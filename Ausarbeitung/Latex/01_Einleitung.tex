%%%%%%%%%%%%%%%%%%%%%%%%%%%%%%%%% Einleitung %%%%%%%%%%%%%%%%%%%%%%%%%%%%%%%%%

\chapter{Einleitung}
\label{chap:Einleitung}

Für die Organisation von Leichtathletikveranstaltungen bedarf es einer strukturierten und vor allem richtigen Zeitplanung, um Unannehmlichkeiten zu vermeiden bzw. einen reibungslosen und professionellen Ablauf der Veranstaltung zu gewährleisten. Momentan werden die Daten für die Erstellung eines Zeitplans der Veranstaltungen der LG Telis Finanz aus einer Teilnehmerliste, in welcher die terminlichen Eckdaten von Seltec (www.seltec.at) bereits eingeplant wurden, extrahiert. Die Teilnehmerliste wird im Folgenden als Ursprungsdatei bzw. Inputfile bezeichnet. Die Informationen der Ursprungsdatei müssen dabei manuell ins \ac{HTML}-Format übertragen werden, um den Zeitplan zu erstellen. Da die Ursprungsdatei mit den relevanten Daten sehr unstrukturiert ist und viele überflüssige Informationen enthält, ist der Zeitplan sehr aufwändig zu generieren. Zudem entstehen durch die redundante Datenhaltung und die manuelle Übertragung ins HTML-Format manchmal inkonsistente Zeitplandaten. \\
Im Rahmen des Projektseminars „Erstellung eines Zeitplan-Generators für Leichtathletikveranstaltungen“ soll deshalb ein Tool entwickelt werden, welches ermöglicht, dass aus der Ursprungsdatei im Seltec-Format automatisch ein strukturierter Zeitplan in HTML-Form erstellt wird. Das Tool verarbeitet den Inhalt der Ursprungsdatei und visualisiert die relevanten Veranstaltungsdaten. Das Datum der Veranstaltung, die Uhrzeit des Wettkampfbeginns, die Altersklasse (offizielle Bezeichnung) und die Disziplin bzw. Wettkampfart (mit allgemeingültigen Abkürzungen) sind relevante Veranstaltungsdaten und sollen Bestandteil der gewünschten Ergebnistabelle sein. Ziel ist es, für jeden Wettkampftag eine separate Tabelle zu erzeugen, wobei das jeweilige Datum als Überschrift festgelegt wird (in Anlehnung an den Zeitplan der Sparkassen Gala 2016) \cite{gala}.\protect{\footnote{\url{http://www.sparkassen-gala.de/zeitplan.php}, zuletzt abgerufen am 21.11.16}}
Darüber hinaus soll für den Zeitplan-Generator eine Desktop-App entwickelt werden.\\
Ein Grund für die Umsetzung des Projektes ist die Richtigkeit und Konsistenz der Daten bei einer automatischen Generierung. Das Tool erkennt alle relevanten Daten maschinell, was einen Datenverlust ausschließt und einen vollständigen Zeitplan garantiert. Die Fehlerquelle beim manuellen Erstellen des Terminplans wird also eliminiert, da beispielsweise Tippfehler oder falsche bzw. fehlende Eintragungen nicht mehr möglich sind. Dieser Punkt ist für die Leichtathleten bzw. alle Interessierten für Leichtathletikveranstaltungen von essentieller Bedeutung, da an die im Zeitplan angegebenen Terminplaninformationen unter anderem die Anreise, das Training und der Tagesablauf abgestimmt werden. Dies hätte im schlimmsten Fall zur Folge, dass der Sportler zu seinem Wettkampf nicht antreten kann, da er aufgrund einer falschen Information zu spät oder gar nicht zum Wettkampfort angereist ist.\\
Außerdem hat die automatische Generierung eines Zeitplans für Leichtathletikveranstaltungen die positive Auswirkung, dass der Ersteller deutlich weniger Zeit für die Zeitplanerstellung benötigt. Es ist nicht mehr erforderlich mühsam alle relevanten Informationen zu identifizieren und diese anschließend in den Zeitplan zu übertragen, da dies der Zeitplan-Generator komplett übernimmt. Der Nutzer muss lediglich das gewünschte Inputfile auswählen, um einen fertigen und übersichtlichen Zeitplan zu erhalten. \\
Des Weiteren ermöglicht der Zeitplan-Generator eine einheitliche Darstellung von Leichtathletikzeitplänen. Alle Terminpläne für die jeweiligen Veranstaltungen werden nach dem gleichem Schema und Design erzeugt, was es für den Nutzer leichter macht, sich zurecht zu finden. Bei der Erstellung werden immer die offiziellen Altersklassen bzw. Kategorien der Leichtathletikwettbewerbe berücksichtigt. Zudem muss sich der Nutzer nicht bei jeder neuen Veranstaltung auf unterschiedliche Namenskonventionen bzw. eine andere Tabellendarstellung umstellen, da die Daten immer gleich strukturiert sind und beispielsweise keine Teilnehmerklassen zusammengefasst werden. Dies führt dazu, dass unnötige Missverständnisse vermieden werden und die Benutzerfreundlichkeit erhöht wird.\\
Ein weiterer Vorteil bei der automatischen Erstellung eines Zeitplans ist, dass das Tool vielseitig verwendet werden kann. Als Inputfile wird lediglich ein HTML-File benötigt, dass in einer bestimmten Form die relevanten Daten für die Terminplanerstellung enthält. Eine Möglichkeit ist die Kompatibilität des Zeitplan-Generators mit der Ursprungsdatei von Seltec zu nutzen. Aber auch für kleinere Vereine ist es leicht möglich, einen Terminplan zu generieren, der online verfügbar ist und den allgemeinen Standards entspricht. Es wird lediglich eine Ursprungsdatei mit einem bestimmten Aufbau benötigt.\\
Es gibt also viele Gründe, die eine Entwicklung des Zeitplan-Generators für Leichtathletikveranstaltungen favorisieren. Besonders die Strukturiertheit und Richtigkeit sprechen für die Verwirklichung des Projektes. Im Folgenden wird nun auf das genaue Vorgehen bei der Entwicklung des Zeitplan-Generators eingegangen. Dabei wird zuerst auf die Projektdurchführung und die Konzeption des Tools eingegangen. Dies beinhaltet die verwendeten Hilfsmittel zur Durchführung des Projektes, die Spezifikation der Ursprungsdatei und die Aufteilung der Entwicklungen in Pakete. Im darauffolgenden Schritt werden die Programmarchitektur und der konkrete Ablauf bei der Entwicklung des Zeitplan-Generators erklärt. In diesem Teil werden die einzelnen HTML-Files zur Erstellung des Zeitplan-Generators und die Erstellung einer Desktop-App näher erläutert. In den abschließenden Punkten der Arbeit wird das Projekt evaluiert und das Ergebnis präsentiert.


%%%%%%%%%%%%%%%%%%%%%%%%%%%%%%%%%%%%%%%%%%%%%%%%%%%%%%%%%%%%%%%%%%%%%%%%%%%%%%
