%%%%%%%%%%%%%%%%%%%%%%%%%%%%%%%%% Einleitung %%%%%%%%%%%%%%%%%%%%%%%%%%%%%%%%%

\chapter{Einleitung}
\label{chap:Einleitung}

Das richtige Zeitmanagement ist für jeden eine Herausforderung, da es oft nicht einfach ist alle Termine in einer begrenzten Zeit zu managen. Deshalb ist es wichtig Termine bzw. Veranstaltungen zu planen. Auch für die Organisation von Leichtathletikveranstaltungen bedarf es einer strukturierten und vor allem richtigen Zeitplanung, um Unannehmlichkeiten zu vermeiden bzw. einen reibungslosen und professionellen Ablauf der Veranstaltung zu gewährleisten. Momentan werden die Zeitpläne der Veranstaltungen der LG Telis Finanz manuell aus einem Ursprungsfile generiert. Da das Dokument mit den relevanten Daten sehr unstrukturiert ist und viele überflüssige Informationen enthält, ist der Zeitplan natürlich sehr aufwändig zu erstellen. Zudem ist bei einer manuellen Erstellung die Möglichkeit eines fehlerhaften Zeitplans immer gegeben.

Im Rahmen des Projektseminars „Erstellung eines Zeitplan-Generators für Leichtathletikveranstaltungen“ soll deshalb ein Tool entwickelt werden, welches ermöglicht, dass aus dem \ac{HTML}-Ursprungsfile einer bestimmten Form automatisch ein strukturierter Zeitplan in HTML-Form erstellt wird. Das Tool erkennt alle benötigten Daten in dem Eingangsfile, die für die Erstellung des Zeitplans notwendig sind und generiert daraus eine Tabelle. Dabei sollen das Datum der Veranstaltung, die Uhrzeit des Wettkampfbeginns, die Altersklasse (offizielle Bezeichnung) mit dem zugehörigen Geschlecht und die Wettkampfart (mit allgemeingültigen Abkürzungen) Bestandteil der gewünschten Ergebnistabelle sein. Ziel ist es für jeden Wettkampftag eine separate Tabelle zu erzeugen, weshalb zuerst nach dem Datum sortiert und gruppiert werden muss. Gleichzeitig soll das Datum als Überschrift für die jeweiligen Zeitpläne aufgenommen werden. Der Zeitplan selbst soll folgendermaßen aufgebaut sein: Der Beginn der Veranstaltung steht in der ersten Spalte unter dem Titel Zeit, wobei aufsteigend nach der Uhrzeit sortiert wird; die Altersklassen werden in der ersten Zeile ab der zweiten Zelle angeführt, wobei diese aszendierend und nach dem Geschlecht sortiert dargestellt werden und die Wettkampfbezeichnung wird schließlich in der richtigen Zelle ausgehend von Datum, Uhrzeit und Altersklasse eingetragen. Für die Ergebnistabelle dient als Referenz der Zeitplan der Sparkassen Gala 2016.\protect{\footnote{\url{http://www.sparkassen-gala.de/zeitplan.php}, abgerufen am 21.11.16}}
Die Herausforderung besteht in der Korrektheit, Kompaktheit und Übersichtlichkeit der Tabelle.

Ein sehr motivierender Grund für die Umsetzung des Projektes ist die Richtigkeit und Konsistenz der Daten bei einer automatischen Generierung. Das Tool erkennt alle relevanten Daten maschinell, was einen Datenverlust ausschließt und einen vollständigen Zeitplan garantiert. Die Fehlerquelle beim manuellen Erstellen des Terminplans wird also eliminiert, da beispielsweise keine Tippfehler oder falsche bzw. fehlende Eintragungen mehr möglich sind. Dieser Punkt ist für die Leichtathleten bzw. alle Interessierten für Leichtathletikveranstaltungen von essentieller Bedeutung, da aufgrund falscher Terminplaninformationen die Anreise, das Training und der Tagesablauf beeinträchtigt werden. Diese Punkte werden natürlich an den Termin der Veranstaltung ausgerichtet und im schlimmsten Fall hätte das zur Folge, dass der Sportler zu seinem Wettkampf nicht antreten kann, da er aufgrund einer falschen Information zu spät oder gar nicht zum Wettkampfort angereist ist. Das würde natürlich ein schlechtes Licht auf die Veranstaltung bzw. den Veranstalter werfen.
Außerdem hat die automatische Generierung eines Zeitplans für Leichtathletikveranstaltungen die positive Auswirkung, dass beispielsweise der Ersteller deutlich weniger Zeit für die Zeitplanerstellung benötigt. Es ist nicht mehr nötig mühsam alle relevanten Informationen zu identifizieren und diese anschließend in den Zeitplan zu übertragen, da dies der Zeitplan-Generator komplett übernimmt. Der Nutzer muss lediglich das gewünschte Eingangsfile auswählen, um einen fertigen und übersichtlichen Zeitplan zu erhalten. 
Des Weiteren ermöglicht der Zeitplan-Generator eine einheitliche Darstellung von Leichtathletikzeitplänen. Alle Terminpläne für die jeweiligen Veranstaltungen werden nach dem gleichen Schema erzeugt, was es für den Nutzer natürlich auch leichter macht sich zurecht zu finden. Bei der Erstellung werden immer die offiziellen Altersklassen bzw. Kategorien der Leichtathletikwettbewerbe berücksichtigt. Zudem muss sich der Nutzer nicht bei jeder neuen Veranstaltung auf unterschiedliche Namenskonventionen bzw. eine andere Tabellendarstellung umstellen, da die Daten immer gleich strukturiert sind und beispielsweise keine Teilnehmerklassen zusammengefasst werden. Dies führt natürlich wieder dazu, dass unnötige Missverständnisse vermieden werden und die Benutzerfreundlichkeit erhöht wird.
Ein weiterer Vorteil bei der automatischen Erstellung eines Zeitplans ist, dass das Tool vielseitig verwendet werden kann. Als Input wird lediglich ein HTML-File benötigt, dass in einer bestimmten Form die relevanten Daten für die Terminplanerstellung enthält. Dies macht es beispielsweise auch für kleinere Verbände leicht möglich einen Terminplan zu generieren, der online verfügbar ist und den allgemeinen Standards entspricht.

Es gibt also viele gute Gründe, die die Entwicklung eines Zeitplan-Generators für Leichtathletikveranstaltungen rechtfertigen. Besonders die Strukturiertheit und Richtigkeit sprechen für die Verwirklichung des Projektes. Im Folgenden wird nun auf das genaue Vorgehen bei der Entwicklung des Zeitplan-Generators eingegangen.



%%%%%%%%%%%%%%%%%%%%%%%%%%%%%%%%%%%%%%%%%%%%%%%%%%%%%%%%%%%%%%%%%%%%%%%%%%%%%%
