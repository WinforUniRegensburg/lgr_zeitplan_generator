%%%%%%%%%%%%%%%%%%%%%%%%%%%%%%%%% Einleitung %%%%%%%%%%%%%%%%%%%%%%%%%%%%%%%%%

\chapter{Einleitung}
\label{chap:Einleitung}

Für die Organisation von Leichtathletikveranstaltungen bedarf es einer strukturierten und vor allem richtigen Zeitplanung, um Unannehmlichkeiten zu vermeiden bzw. einen reibungslosen und professionellen Ablauf der Veranstaltung zu gewährleisten. Momentan erhält man die Daten für die Erstellung eines Zeitplans der Veranstaltungen der LG Telis Finanz aus einer Teilnehmerliste oder aus einem parallel gepflegten Zeitplandokument. Dabei müssen die Informationen des jeweiligen Dokuments manuell ins \ac{HTML}-Format übertragen werden, um den Zeitplan zu erstellen. Da die Teilnehmerliste mit den relevanten Daten sehr unstrukturiert ist und viele überflüssige Informationen enthält, ist der Zeitplan natürlich sehr aufwändig zu generieren. Zudem entstehen durch die redundante Datenhaltung und die manuelle Übertragung ins HTML-Format manchmal inkonsistente Zeitplandaten. Diese Gefahr besteht bei der Teilnehmerliste und im manuell gepflegten Zeitplandokument. \\
Im Rahmen des Projektseminars „Erstellung eines Zeitplan-Generators für Leichtathletikveranstaltungen“ soll deshalb ein Tool entwickelt werden, welches ermöglicht, dass aus der Teilnehmerliste, welche im Folgenden als \ac{HTML}-Ursprungsdatei bzw. Inputfile bezeichnet wird, einer bestimmten Form automatisch ein strukturierter bzw. tabellarischer Zeitplan in HTML-Form erstellt wird. Das Datum der Veranstaltung, die Uhrzeit des Wettkampfbeginns, die Altersklasse (offizielle Bezeichnung) mit dem zugehörigen Geschlecht und die Disziplin bzw. Wettkampfart (mit allgemeingültigen Abkürzungen) sollen Bestandteil der gewünschten Ergebnistabelle sein. Ziel ist es für jeden Wettkampftag eine separate Tabelle zu erzeugen, die das jeweilige Datum als Überschrift hat (in Anlehnung an den Zeitplan der Sparkassen Gala 2016).\protect{\footnote{\url{http://www.sparkassen-gala.de/zeitplan.php}, abgerufen am 21.11.16}}
Die Herausforderung besteht in der Korrektheit, Kompaktheit und Übersichtlichkeit der Tabelle. 
Darüber hinaus soll für den Zeitplan-Generator eine Desktop-App entwickelt werden. Ein weiteres Ziel ist, dass das Tool soll mit dem Seltec-Format (www.seltec.at) kompatibel ist.\\
Der motivierende Grund für die Umsetzung des Projektes ist die Richtigkeit und Konsistenz der Daten bei einer automatischen Generierung. Das Tool erkennt alle relevanten Daten maschinell, was einen Datenverlust ausschließt und einen vollständigen Zeitplan garantiert. Die Fehlerquelle beim manuellen Erstellen des Terminplans wird also eliminiert, da beispielsweise keine Tippfehler oder falsche bzw. fehlende Eintragungen mehr möglich sind. Dieser Punkt ist für die Leichtathleten bzw. alle Interessierten für Leichtathletikveranstaltungen von essentieller Bedeutung, da auf die im Zeitplan angegebenen Terminplaninformationen beispielsweise die Anreise, das Training und der Tagesablauf abgestimmt werden. Dies hätte im schlimmsten Fall zur Folge, dass der Sportler zu seinem Wettkampf nicht antreten kann, da er aufgrund einer falschen Information zu spät oder gar nicht zum Wettkampfort angereist ist. Das könnte ein schlechtes Licht auf die Veranstaltung bzw. den Veranstalter werfen.\\
Außerdem hat die automatische Generierung eines Zeitplans für Leichtathletikveranstaltungen die positive Auswirkung, dass beispielsweise der Ersteller deutlich weniger Zeit für die Zeitplanerstellung benötigt. Es ist nicht mehr nötig mühsam alle relevanten Informationen zu identifizieren und diese anschließend in den Zeitplan zu übertragen, da dies der Zeitplan-Generator komplett übernimmt. Der Nutzer muss lediglich das gewünschte Inputfile auswählen, um einen fertigen und übersichtlichen Zeitplan zu erhalten. \\
Des Weiteren ermöglicht der Zeitplan-Generator eine einheitliche Darstellung von Leichtathletikzeitplänen. Alle Terminpläne für die jeweiligen Veranstaltungen werden nach dem gleichen klarem Schema und Design erzeugt, was es für den Nutzer natürlich auch leichter macht sich zurecht zu finden. Bei der Erstellung werden immer die offiziellen Altersklassen bzw. Kategorien der Leichtathletikwettbewerbe berücksichtigt. Zudem muss sich der Nutzer nicht bei jeder neuen Veranstaltung auf unterschiedliche Namenskonventionen bzw. eine andere Tabellendarstellung umstellen, da die Daten immer gleich strukturiert sind und beispielsweise keine Teilnehmerklassen zusammengefasst werden. Dies führt natürlich wieder dazu, dass unnötige Missverständnisse vermieden werden und die Benutzerfreundlichkeit erhöht wird.\\
Ein weiterer Vorteil bei der automatischen Erstellung eines Zeitplans ist, dass das Tool vielseitig verwendet werden kann. Als Input wird lediglich ein HTML-File benötigt, dass in einer bestimmten Form die relevanten Daten für die Terminplanerstellung enthält. Dies macht es beispielsweise auch für kleinere Vereine leicht möglich einen Terminplan zu generieren, der online verfügbar ist und den allgemeinen Standards entspricht.\\
Es gibt also viele gute Gründe, die für eine Entwicklung des Zeitplan-Generators für Leichtathletikveranstaltungen sprechen. Besonders die Strukturiertheit und Richtigkeit sprechen für die Verwirklichung des Projektes. Im Folgenden wird nun auf das genaue Vorgehen bei der Entwicklung des Zeitplan-Generators eingegangen. Dabei wird zuerst auf die Projektdurchführung und die Konzeption des Tools eingegangen. Dies beinhaltet die verwendeten Hilfsmittel zur Durchführung des Projektes, die Spezifikation des Rohtextes und die Aufteilung der Entwicklungen in Pakete. Im darauffolgenden Schritt werden die Programmarchitektur und der konkrete Ablauf bei der Entwicklung des Zeitplan-Generators erklärt. In diesem Teil werden die einzelnen HTML-Files zur Erstellung des Zeitplan-Generators und die Erstellung einer Desktop-App näher erläutert. In den abschließenden Punkten der Arbeit wird das Projekt evaluiert und das Ergebnis präsentiert.


%%%%%%%%%%%%%%%%%%%%%%%%%%%%%%%%%%%%%%%%%%%%%%%%%%%%%%%%%%%%%%%%%%%%%%%%%%%%%%
