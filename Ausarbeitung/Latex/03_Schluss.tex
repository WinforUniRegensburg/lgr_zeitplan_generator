%%%%%%%%%%%%%%%%%%%%%%%%%%%%%%%%%% Schluss %%%%%%%%%%%%%%%%%%%%%%%%%%%%%%%%%%%

\chapter{Schluss}
\label{chap:Schluss}

In diesem letzten Kapitel, wird die Erreichung der gesetzten Ziele diskutiert.
Das übergeordnete Ziel, einen Zeitplan automatisch aus einer Teilnehmerliste zu generieren, wurde erreicht. Das Tool ermöglicht es dem Nutzer die Ursprungsdatei einzulesen und erzeugt daraus einen fertigen Zeitplan im HTML-Format. Es sind keine weiteren Zwischenschritte oder zusätzliche Aufgaben des Nutzers nötig, um das gewünschte Ergebnis zu erhalten. Zudem wird dem Nutzer das Exportieren bzw. das lokale Abspeichern des Zeitplans im HTML-Format ermöglicht. Da der Nutzer nur wenige Aktionen zur Verfügung hat und trotzdem ein voll funktionsfähiger Zeitplan generiert wird, ist das Tool auch unkompliziert zu bedienen.\\
Des Weiteren wurde auf eine vollständige, richtige bzw. konsistente, dynamische und kompakte Erzeugung des Zeitplans geachtet. Die Veranstaltungsdaten für den Zeitplan sind vollständig und richtig, da der Zeitplan-Generator alle in der Ursprungsdatei enthaltenen relevanten Daten identifiziert, abspeichert und daraus einen Zeitplan erzeugt. Ein Datenverlust ist also ausgeschlossen. Zudem wird gewährleistet, dass die Informationen eindeutig sind und nur einmal im Zeiplan enthalten sind. Durch das schrittweise Hinzufügen der verschiedenen Veranstaltungen, werden nur die wirklich benötigten Zeilen und Spalten im Zeitplan erzeugt. Der Zeitplan wird also dynamisch generiert. Dies und die Verwendung allgemeiner Abkürzungen für bestimmte Wörter ermöglicht eine kompakte und übersichtliche Darstellung des Zeitplans.\\
Darüber hinaus ist dieser Zeitplan an den Zeitplan der Sparkassen-Gala angelehnt und verwendet allgemeine Bezeichner für die Leichtathletik-Altersklassen. Auf die Zusammenfassung von Leichtathletik-Altersklassen wurde bewusst verzichtet, da die Datenhaltung dadurch besser ist und der Zeitplan übersichtlicher wird.\\
Zudem wurde auf ein ansprechendes Design und die Nutzerfreundlichkeit des Tools geachtet. Das Design ist farblich an die Website der LG Telis Finanz angelehnt. Auf unnötige Elemente wurde bewusst verzichtet. Das Design ist klar und unterstützt eine einfache Bedienung.\\
Das Ziel eine Desktop-App für den Zeitplan-Generator zu erzeugen wurde ebenfalls erreicht. Der Nutzer kann die App auf den Betriebssystemen Windows, Mac OS und Linux installieren und nutzen. Die Desktop-App beinhaltet alle Funktionalitäten. Zudem ist die Desktop-App lokal ausführbar, d.h. sie benötigt keine Internetverbindung zur Erzeugung eines Zeitplans.\\
Als letztes wurde das Ziel, das Tool als Open Source-Software zu veröffentlichen, umgesetzt. Jeder interessierte Nutzer kann das Tool kostenlos herunterladen und in vollem Umfang nutzen. Des Weiteren wurde ein README-File erstellt, das allgemeine Informationen und eine Anleitung für die Installation des Tools enthält.\\
Zusammenfassend wurden alle Projektziele erreicht und es wurden zusätzliche Erweiterungen, wie beispielsweise die Gestaltung eines eigenen Icons für die Desktop-App umgesetzt. Alle Termine und gesetzten Meilensteine, während des Projektes, konnten trotz kleinerer Abstimmungsprobleme eingehalten werden. Das Projekt kann fristgerecht am geplanten Termin abgegeben werden.

%%%%%%%%%%%%%%%%%%%%%%%%%%%%%%%%%%%%%%%%%%%%%%%%%%%%%%%%%%%%%%%%%%%%%%%%%%%%%%
