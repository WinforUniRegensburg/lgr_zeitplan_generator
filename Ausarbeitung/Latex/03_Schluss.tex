%%%%%%%%%%%%%%%%%%%%%%%%%%%%%%%%%% Schluss %%%%%%%%%%%%%%%%%%%%%%%%%%%%%%%%%%%

\chapter{Schluss}
\label{chap:Schluss}

Nach der Fertigstellung des Zeitplan-Generators wird dieser der Öffentlichkeit als Open Source-Software zur Verfügung gestellt. Die Veröffentlichung als Open Source-Variante begünstigt einen höheren Verbreitungs- und Nutzungsgrad des Tools. Zudem ist es möglich, dass jeder den Code des Zeitplan-Generators nachvollziehen bzw. überprüfen kann. 
Das herausgegebene Paket enthält die vier Web-Dokumente FileUploader, ParagraphHandler, Cleaner und GenerateTable, die die Funktionalität des Zeitplan-Generators beinhalten. Des Weiteren sind die für die Desktop-App benötigten Dateien index.html, main.js und package.json Teil des Pakets. Dies beinhaltet die verschiedenen Module von node.js, die jQuery- bzw. Bootstrap-Version und das Icon. \\
\\
Darüber hinaus wird ein README-File erstellt, das den Nutzer über die Desktop-App informiert. Das File erklärt zunächst, wo der Zeitplan-Generator eingesetzt werden kann und für wen die Nutzung interessant ist. Der Zeitplan-Generator erzeugt einen Zeitplan für Leichtathletikveranstaltungen aus einer bestimmten Ursprungsdatei. Unter dem Stichpunkt Requirements wird die benötigte Struktur und das Format der Ursprungsdatei erläutert (vgl. 2.1.2). Zudem ist das Tool kompatibel, wenn der Nutzer bisher seinen Zeitplan mit dem Tool von Seltec (www.seltec.at) erstellt hat. Möchte der Nutzer also einen Zeitplan für Leichtathletikveranstaltungen erzeugen und hat eine Ursprungsdatei dieser Form vorliegen bzw. verwendet bisher das Tool von Seltec, ist die Nutzung des Tools potenziell möglich.\\
Es wird erklärt, wie der Zeitplan-Generator genutzt werden kann. Eine Möglichkeit ist, das Tool über die Datei index.html im Browser zu starten. Es wird die Verwendung der Browser Google Chrome oder Firefox empfohlen. 
Die zweite Möglichkeit ist den Zeitplan-Generator als Desktop-App zu nutzen. Für die Nutzung der Desktop-App ist es nötig, dass der Nutzer Node.js und Electron installiert. Die jeweiligen Download-Links sind ebenfalls in dieser Beschreibung vorhanden. Für die Installation der Desktop-App wird die genaue Anleitung gegeben. In der Kommandozeile wird durch den Aufruf der jeweiligen Befehle für die Betriebssysteme im Verzeichnis der App diese selbst installiert (vgl. 2.2.6). Für eventuelle Schwierigkeiten bei der Installation werden zusätzlich einige weiterführende Links angegeben, die bei der Problemlösung helfen.\\
Weitere Bestandteile des README-Files sind die Auflistung der verwendeten Skriptsprachen, Software und Frameworks, sowie die Beschreibung des Outputs des Zeitplan-Generators. Das Tool verwendet HTML5, JavaScript, jQuery, Bootstrap, Node.js und Electron. Als Output wird eine Tabelle im HTML-Format auf dem Display ausgegeben, die der Nutzer für eine weitere Verwendung am Computer abspeichern kann.\\
\\
Der Zeitplan-Generator wird als Open Source-Software mit allen erzielten Ergebnissen und Inhalten veröffentlicht. Zudem wird dem Nutzer eine Anleitung mit dem README-File mitgeliefert. Das Tool kann von jedem Interessierten kostenlos und in vollem Umfang genutzt werden.



%%%%%%%%%%%%%%%%%%%%%%%%%%%%%%%%%%%%%%%%%%%%%%%%%%%%%%%%%%%%%%%%%%%%%%%%%%%%%%
